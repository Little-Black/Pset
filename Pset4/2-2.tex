\documentclass{6046}

\author{Lingfu Zhang}
\problem{2-2}
\collab{none}

\begin{document}

\paragraph{(a)}
The input of this problem are 
a key $k$ and two B-trees 
$T_1$ and $T_2$ with same minimum degree parameter $t$ 
and same height $h_1=h_2$. 
All keys in $T_1$ are strictly smaller than $k$
 while all keys in $T_2$ are strictly larger than $k$. 
The out put should be another 
B-tree with minimum degree parameter $t$ 
and contains exactly all 
keys in $T_1$ and $T_2$ plus $k$
\\

Let the root node of $T_1$ and $T_2$ 
be $r_1$ and $r_2$, respectively. 
Then we create a node $r$, 
whose keys are keys in $r_1$, 
$k$, 
and keys in $r_2$, in that order. 
The children of $r$ are children of $r_1$ and 
children of $r_2$, in their previous order, 
and the children of $r_2$ come after 
the children of $r_1$. 
\\
Now if $r.n \leq 2t-1$, return $r$; 
otherwise, create nodes 
$r_1'$, $r_2'$ and $r'$, 
where the keys in $r_1'$ are the first half in r:   
$$\{ r.key_1, ... , r.key_{[\frac{r.n-1}{2}]} \}$$
and the keys in $r_2'$ are  
$$\{ r.key_{[\frac{r.n+3}{2}]}, ... , r.key_{[r.n]} \}$$
The children of $r_1'$ are $r.c_1$, ... ,  
$r_1'$ is $r.c_{[\frac{r.n+1}{2}]}$;  
the children of $r_2'$ are $r.c_{[\frac{r.n+3}{2}]}$, ... , 
$r.c_{r.n}$. 
Let $r'$ has only one key $r.key_{[\frac{r.n+1}{2}]}$, 
and $r'.c_1 = r_1'$ and $r'.c_2 = r_2'$. 
Then return $r'$. 
\\

The tree starting from $r$ we create in this algorithm 
contains exactly all keys 
in $T_1$, $T_2$, and $k$, 
and as all its nodes except for the root node 
are from either $T_1$ or $T_2$, 
the number of keys they contain is between 
$t-1$ and $2t-1$. 
As $T_1$ and $T_2$ has same height, 
$r$ leads a B-tree except for that 
the $r.n$ might exceed $2t-1$. 
If $r.n \leq 2t-1$, $r$ itself is qualified. 
However, if $r.n > 2t-1$, as 
both $r_1.n \leq 2t-1$ and $r_2.n \leq 2t-1$, 
we have that $r.n \leq 4t-1$. 
Thus both $r_1'.n$ and $r_2'.n$ 
are $\leq 2t-1$ and $> t-1$. 
As $r'$ contains exactly all keys in $r$, 
and all other nodes in $r'$ are from $T_1$ or $T_2$, 
we might say that $r'$ is a qualified tree. 
\\

Making $r$, $r_1'$, $r_2'$ and $r'$ all takes 
$O(t)$ time, and this time is considered constant. 

\paragraph{(b)}
The input of this problem are 
a key $k$ and two B-trees 
$T_1$ and $T_2$ with same minimum degree parameter $t$ 
and have heights $h_1$ and $h_2$, where $h_1 = h_2+1$. 
All keys in $T_1$ are strictly smaller than $k$
 while all keys in $T_2$ are strictly larger than $k$. 
The out put should be another 
B-tree with minimum degree parameter $t$ 
and contains exactly all 
keys in $T_1$ and $T_2$ plus $k$
\\

Let the roots of $T_1$ and $T_2$ are 
$r_1$ and $r_2$ respectively. 
Let the last child of $r_1$ be $r_l.lc$. 
Then $r_1.lc$ has the same height as $r_2$. 
We do the same algorithm in (a) on 
$r_1.lc$ and $r_2$, 
and assume the result is $r_g$. 
If the tree leading by $r_g$ has height $h_1-1$, 
this indicates $r_g.n = r_1.lc + r_2 > t-1$, 
and $r_g.n < 2t-1$. 
We can replace $r_1.lc$ with $r_g$, and return $r_1$. 
\\
On the other hand, 
if the tree leading by $r_g$ has height $h_1$, 
this indicates that 
$r_g.n = 1$. 
Add a key, which equals $r_g.key_1$, to $r$, 
and that key should be the largest key of $r$. 
Let the two nodes before and after that key 
be the two child of $r_g$ 
(thus $r_1.lc$ is overwritten).
Now return $r_1$ if $r_1.n \leq 2t-1$; 
otherwise, do the same split that we did on $r$ in (a), 
and return the tree we get. 
\\

According to (a), 
the $r_g$ we get here is a qualified B-tree 
with all keys from the tree under $r_1.lc$ and $T_2$. 
If it has height $h_1-1$, or $h_2$, 
replacing $r_1.lc$ with $r_g$ simply 
adds all keys in $T_2$. 
As $r_g.n$ is within the range, $r_1$ is what we need. 
If $r_g$ has height $h_1$, 
the process we take also 
deletes $r_1.lc$ from the tree leading by $r_1$ and 
adds $r_g$, 
and the number of keys in all nodes 
are between $t-1$ and $2t-1$, 
except for $r_1$, which might have $2t$ keys. 
This can be handled using the same approach we dealt with 
$r$ in (a). 
\\

Getting $r_g$ requires constant time, according to (a). 
Merging $r_g$ with $r_1$ under both conditions 
also takes constant time, 
as we need at most add one key to $r_1$ and redirect two children. 
The possible splitting of $r_1$ 
still takes constant time, 
according to (a). 
Thus the total time is constant. 
\\

\paragraph{(c)}
The insertion takes an augmented B-tree and 
a key $k$ not in the tree as input, 
returns another augmented B-tree that 
contains all keys in the input B-tree and $k$; 
the deletion takes an augmented B-tree and 
a key $k$ that is in the tree as input, 
returns a B-tree that contains all keys in 
the input tree except for $k$. 
All B-trees here have the same minimum degree $t$. 
\\

Both algorithms are similiar to 
the algorithms for non augmented B-tree. 
We let $x.h$ be the height of the subtree below $x$. 
For the insertion, 
the "B-TREE-SPLIT-CHILD($x$, $i$)" 
in CLRS 18.2 should be modified.  
We need to let $z.h = y.h$ after creating 
$z$, as the subtree below $y$ 
will be split to two below 
$y$ and $z$, respectively. 
Besides, 
for "B-TREE-INSERT($T$, $i$)", 
we should let $s.h = r.h +1$ when 
creating $s$.  
\\

As at most one $s$ will be created 
during one insertion, 
and split at most $h$ times, 
the time added is $O(h)$, 
and the total time is $O(h)$. 
\\

For deletion, 
like in CLRS 18.3, we first 
check whether the root node is empty; 
if empty, we let its only child 
be the root instead. 
Then call a function 
"DELETE-NON-MINIMUM($x$, $k$)", 
which tries to delete 
$k$ from the subtree below $x$, 
on condition that $k$ is in the subtree, 
and $x.n$ is not minimum (for non root is $t-1$, root is $0$). 
In "DELETE-NON-MINIMUM", 
we first check if $k$ is a key in $x$:  
if true, check the child before and after $k$: 
if the number of keys in either doesn't reach the minimum, 
find the predecessor or successor of $k$, 
and run
"DELETE-NON-MINIMUM($child\ before\ k$, $k.predecessor$)" 
or "DELETE-NON-MINIMUM($child\ after\ k$, $k.successor$)". 
Then replace $k$ with $k.predecessor$ or $k.successor$. 
If both reaches the minimum ($t-1$), 
merge the two children to one node, 
which has the same height as the two children, 
and keys from both and $k$. 
Then delete $k$ from the new node. 
\\
If $k$ is not a key in $x$, 
find the child $x.c_j$ of $x$ that contains $k$. 
If $x.c_j.n > t-1$, 
just call "DELETE-NON-MINIMUM($x.c_j$, $k$)"; 
otherwise, 
check 
$x.c_{j-1}.n$ and $x.c_{j+1}$. 
If either is $>t-1$, 
add $x.key_j$/$x.key_{j+1}$ to $x.c_j$, 
move the last/first key of 
$x.c_{j-1}$/$x.c_{j+1}$ to 
$x.key_j$/$x.key_{j+1}$, 
move the last/first subtree  of 
$x.c_{j-1}$/$x.c_{j+1}$ (if they are not leaves) to 
$x.c_j$. 
Then call "DELETE-NON-MINIMUM($x.c_j$, $k$)". 
If both 
$x.c_{j-1}.n$ and $x.c_{j+1}$ 
$=t-1$, 
merge $x.c_j$ with $x.c_{j-1}$, 
creating a new node with the same height as 
$x.c_j$, 
and delete $k$ from the new node. 
\\

If $k$ is at a leaf, 
DELETE-NON-MINIMUM is called $h$ times, 
while at every level at most one merge would happen and takes constant time; 
otherwise, as searching for sucessor or predecessor would be needed, 
which takes $O(h)$ time. 
\\

\paragraph{(d)}

The input of this problem are 
a key $k$ and two B-trees 
$T_1$ and $T_2$ with same minimum degree parameter $t$ 
and height $h_1$ and $h_2$. 
All keys in $T_1$ are strictly smaller than $k$
 while all keys in $T_2$ are strictly larger than $k$. 
The out put should be another 
B-tree with minimum degree parameter $t$ 
and contains exactly all 
keys in $T_1$ and $T_2$ plus $k$
\\

Let's define "COMBINE($r_1$, $r_2$, $k$)", 
which takes two augemented B-tree root 
$r_1$, and $r_2$, and $k$, 
given that all keys in subtree below $r_1$ is 
strictly smaller than $k$ and 
$k$ is strictly smaller than all keys in subtree below $r_2$, 
and both $r_1.n$ and $r_2.n$ $<2t-1$. 
If $r_1.h = r_2.h$, 
do as (a); 
if $r_1.h > r_2.h$, 
we find the largest node in the subtree below $r_1$ with height 
$r_2.h$, and let it be $r_1'$. 
Then conbine $r_1'$ and $r_2$, 
using the algorithm in (a). 
If the result, $r_r$ has height $r_2.h$, 
replace $r_1'$ with $r_r$, and 
return $r_1$. 
If $r_r$ has height $r_2.h+1$, 
add the only key in $r_r$ 
to $r_1$'s parent node, 
and add the two children of $r_r$ at the same time 
(thus overwrite $r_1'$). 
Now if $r_1'$'s parent has more than $2t-1$ keys, 
split it and add a key to its parent, 
then split that node if it has $2t$ keys, ..., 
until the tree below $r_1$ is a valid B-tree. 
(Note that $r_1$ itself might be split, 
if so we let $r_1$ be its parent, 
namely, the new root node with only one key). 
Return $r_1$. 
For $r_1.h < r_2.h$, 
the approach is the almost the same. 
\\

When $r_1.h > r_2.h$, 
finding $r_1'$ doesn't make any change to both trees. 
According to (a), 
the result of conbining $r_1'$ and $r_2$ is a valid B-tree 
which contains all keys 
from $r_1'$, $r_2$ and $k$. 
If $r_r.h = r_2.h$, then $r_r.n\leq t-1$, 
replacing $r_1'$ with $r_r$ 
makes $r_1$ a valid B-tree with all keys needed. 
If $r_r.h$ is one larger than $r_2.h$, 
then $r_r$ should be added to a higher level. 
After the adding, 
$r_1$ is already a valid B-tree except for one node, 
which might have $2t$ keys. 
Splitting upward from that node, we can get a valid B-tree, 
which in this process the set of all keys under $r_1$ 
is not changed. 
The same for $r_1.h < r_2.h$. 
\\

Searching for $r_1'$ takes $O(|h_1 - h_2|)$ time
(if $r_1.h \neq r_2.h$), 
conbining $r_1'$ with $r_2$ takes constant time as in (a), 
and spliting $O(|h_1 - h_2|)$ times takes 
$O(|h_1 - h_2|)$ time. 
Considering when $h_1 = h_2$ constant time is needed, 
the overall time needed is 
$O(|h_1 - h_2| + 1)$. 



\end{document}

