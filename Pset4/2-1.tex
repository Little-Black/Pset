\documentclass{6046}


\author{Lingfu Zhang}
\problem{2-1}
\collab{none}

\begin{document}

\paragraph{(a)}
The input of this problem are two strings 
$S$ and $P$, with length $n$ and $m$, respectively.  
$S$ contains only $a$ and $b$, 
and $P$ contains only $a$, $b$ and $*$. 
We need to output a sorted integer listi $M$,  
which are indexs $j$ such that 
the continuing substring of $S$ 
with length $m$
starting from $S[j]$ 
matches $P$, 
under conditions that $*$ can match either $a$ and $b$. 
\\

Naively we compare every contunuing substring with length $m$ of $S$, 
from the one starting from $S[0]$ to the one starting from $S[n-m]$. 
It's obviously correct, and if we do this in the order, 
the list $M$ we get is surely sorted. 
\\

As there are $n-m+1$ such substrings, and every comparing 
contains at most $m$ compares between characters. 
The total time is $O(mn)$

\paragraph{(b)}
Same problem as (a). 
\\

We first convert $S$ and $P$ to polynomials: 
$$S(x) = \sum\limits_{i=0}^{n-1} s_ix^i$$
where $s_i = 1$ if $S[i] = ` a\textrm'$, 
and $s_i = \i$ if $S[i] = `b\textrm'$. 
$$P(x) = \sum\limits_{i=0}^{m-1} p_ix^i$$
where $p_i = 1$ if $P[m-1-i] = `a\textrm'$, 
$p_i = \i$ if $P[m-1-i] = `b\textrm'$, 
and $p_i = 0$ if $P[m-1-i] = `*\textrm'$. 
Then compute $R(x) = P(x)S(x)$. 
For $i$ from $0$ to $n-m$, 
if the coefficient of $x^{i+m-1}$ in $R$ has no imaginary part, 
add $i$ to $M$. 
\\

The proof is simple. 
First, we can have 
$$R(x) = \sum\limits_{i=0}^{n+m-2}\sum\limits_{j=0}^{m-1}s_{i-j}p_jx^{i}$$
Here we assume that $s_{k} = 0$ for $k<0$ or $k>n-1$. 
Then 
$s_{i-j}p_j = \i$
if and only if 
$S[i-j]$ and $P[m-1-j]$ are exactly $`a$' and $`b$' or `$b$' and `$a$', 
which indicates that 
$P$ doesn't match the continuing substring of $S$ 
with length $m$ and starting from $S[i-m+1]$. 
On the other side, 
if 
$\displaystyle{\sum\limits_{j=0}^{m-1}s_{i-j}p_j}$ has no imaginary part, 
none of 
$s_{i-j}p_j$ 
is $\i$, and the two strings match. 
As we chack the coefficients of $R$ one by one, 
$M$ must be sorted. 
\\

Computing $R(x) = S(x)P(x)$ takes $O(mn)$ time, 
other operations all take linear time. 
Thus this algorithm takes 
$O(mn)$ time. 
\\

For the example given, 
we have 
$$S(x) = 1 + \i x + x^2 + \i x^3 + \i x^4 + x^5 + \i x^6$$
$$P(x) = \i x + x^2$$
Then we can get
$$R(x) = S(x)P(x) = \i x + 2\i x^3 + (-1-\i)x^5 + 2\i x^6 + \i x^8$$
Examing coefficients of $x^2$, $x^3$, $x^4$, $x^5$ and $x^6$, 
we find only the ones of $x^2$ and $x^4$ have no imaginary part. 
Then $M = \{ 0, 2 \}$. 
\\

\paragraph{(c)}
Let $k$ be the least 2 power no less than $m+n$. 
Then using FFT we need  
to convert $S(x)$ and $P(x)$ to samples, 
computing $R(x) = S(x)P(x)$ 
and converting $R(x)$ back. 
The whole process needs to treat 
$R(x)$, $P(x)$ and $S(x)$ as polinomials with degree $k$. 
Then $O(k\log{k}) = O((m+n)\log{(m+n)})$ time is needed. 
Considering $m \ll n$, 
the time taken is $O(n\log{n})$. 
\\

\paragraph{(d)}
The problem is basically the same as (a), 
except that characters in $D$ and $P$ can be $A$, $G$, $C$,$T$ 
and $*$ for $P$. 
\\

This time we make two pairs of polynomials:
$$D'(x) = \sum\limits_{i=0}^{n-1} d'_ix^i$$
where $d'_i = 1$ if $D[i] = ` A\textrm'$ or `$G$', 
and $d'_i = \i$ if $D[i] = `C\textrm'$ or `$T$'. 
$$D''(x) = \sum\limits_{i=0}^{n-1} s''_ix^i$$
where $d'_i = 1$ if $D[i] = ` A\textrm'$ or `$C$', 
and $d'_i = \i$ if $D[i] = `G\textrm'$ or `$T$'. 

$$P'(x) = \sum\limits_{i=0}^{m-1} p'_ix^i$$
where $p'_i = 1$ if $P[m-1-i] = `A\textrm'$ or `$G$', 
$p'_i = \i$ if $P[m-1-i] = `C\textrm'$ or `$T$', 
and $p'_i = 0$ if $P[m-1-i] = `*\textrm'$. 
$$P''(x) = \sum\limits_{i=0}^{m-1} p''_ix^i$$
where $p''_i = 1$ if $P[m-1-i] = `A\textrm'$ or `$C$', 
$p''_i = \i$ if $P[m-1-i] = `G\textrm'$ or `$T$', 
and $p''_i = 0$ if $P[m-1-i] = `*\textrm'$. 

Then compute $R'(x) = P'(x)S'(x)$ and $R''(x) = P''(x)S''(x)$. 

Similiarly, we need to add $i$ to $M$ 
if coefficients of $x^{i + m - 1}$ 
in both $R'$ and $R''$ 
has no imaginary part. 
\\

To prove that this algorithm works, 
we first see that if $P[i]$ and $D[j]$ 
doesn't match, 
at least one of $p'_{m-1-i}d'_j$ and $p''_{m-1-i}d''_j$ is $\i$. 
Then similiar to (b), 

$P$ matches the continuing substring of $D$ 
with length $m$ and starting from $D[i]$ 
if and only if 
the coefficients of $x^{i+m-1}$ in both
$R'(x)$ and $R''(x)$ has no imaginary part. 
As we chack the coefficients of $R'$ and $R''$ one by one, 
$M$ must be sorted. 
\\

The time used here is twice as much as that in (b), 
thus is $O(mn)$ (not using FFT) or $O(n\log{n})$ (using FFT). 

\end{document}

