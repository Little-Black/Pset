\documentclass{6046}

\author{Lingfu Zhang}
\problem{1-1}
\collab{none}

\begin{document}

\paragraph{(a)}
Consider the situation of $V = {u_0, u_1, u_2}$ and $E = {(u_0, u_1), (u_1, u_2)}$, with 
$p_0 = 2$, $p_1 = 3$, $p_2 = 2$. 
Using the ``greedy'' algorithm described in the problem, 
we will choose $u_1$ at the first step, 
and remove $u_1$, $u_0$ and $u_2$, 
as $u_0$ and $u_2$ are neighbors of $u_1$. 
Then the total profit we get is $3$. 
However, if we select $u_0$ and $u_2$ instead, 
we can get a total profit of $4$, which indicates that the 
``greedy'' algorithm doesn't work. 

\paragraph{(b)}
This problem is that given a tree every vertex of whom has a weight related to it, 
we need to find a set of vertices with maximum total weight 
and no two of selected vertices are adjacent. 
\\
Let's randomly select a vertex $u_0$ as the root of the tree. 
For a given vertex $u_i$, there is exactly one path from $u_i$ to $u_0$. 
\\

Then give some definitions: 
\\
Let's call the next vertex on the path from $u_i$ to $u_0$ the ``Father'' of $u_i$. 
Then every vertex in $V$, except for $u_0$ itself, has a unique ``Father''. 
\\
For two vertices $u_a$ and $u_b$, 
if $u_a$ is the ``Father'' of $u_b$, we call $u_b$ a ``Child'' of $u_a$. 
Then any adjacent vertex of $u_i$ is either its ``Father'' or ``Child''. 
(Otherwise the paths from both vertices to $u_0$ don't contain each other, 
and we can get a circle.
)
\\
Let $G_i$ be a subgraph of $G$, 
such that $G_i$ consists of all $u_j$ that $u_i$ is on the path from $u_j$ to $u_0$, 
and all edges among these vertices. 
Then $G_0 = G$, and for any $i$, $u_i \in G_i$. 
$G_i$ contains all children of $u_i$. 
\\
Let $A[i]$ be the maximum total weight of $G_i$ (with no two adjacent vertices selected), 
and $N[i]$ be the maximum total weight of $G_i$ 
while $u_i$ itself is not selected (with no two adjacent vertices selected). 
We now have $n$ subproblems (of getting $A[i]$ and $N[i]$). 
Then we can calculate $A[i]$ and $N[i]$ by: 
$$N[i] = \sum_{u_j\ is\ child\ of u_i} A[j] $$
$$A[i] = \max \left( p_i + \sum_{u_j\ is\ child\ of\ u_i} N[j],\ N[i]\right)$$
The reason of doing this is that any $G_i$ 
can be devided into many $G_j$ and $u_i$, where $u_j$ are all children of $u_i$
($u_i$ might also have no child at all). 
When calculating $N[i]$, $u_i$ itself is not selected, 
so we have the freedom of selecting the children of $u_i$. 
As all the $G_j$ do not influence each other (there are no edges connecting them), 
we know that $N[i]$ is simply the sum of all $A[j]$. 
\\
When calculating $A[i]$, there are mainly two situations: 
$u_i$ is selected or not. 
If $u_i$ is not selected, the result is simply $N[i]$; 
if $u_i$ is selected, all its children cannot be selected, 
then the maximum total weight should be the sum of all $N[j]$ adds $p_i$. 
$A[i]$ should be the maximum value of these two. 
\\
We shall initialize a list $Father[i]$ to all $-1$.
Then define a function ``calculateValues($i$)'', 
which first get the list of all adjacent vertices of $u_i$. 
If the list has no element other than $Father[i]$, let $A[i] = p_i$ and $N[i] = 0$, then return; else, 
for every $u_j$ in the list, if $j\neq Father[i]$: let 
$Father[j] = i$, and do ``calculateValues($j$)''. 
Finally, calculate $A[i]$ and $N[i]$ using all $A[j]$ and $N[j]$. 
\\
We directly call ``calculateValues($0$)''. 
Then it will call ``calculateValues'' for all children of $u_i$, 
then the all grandchildren of $u_i$, etc. 
As this graph is connected, all vertices will be called, and exactly once, 
because every vertex, except for $u_0$, has exactly one Father. 
\\
For every subproblem, (every calling of ``calculateValues''), 
it does some addings and a comparing. 
However, as every vertex has one Father, every 
$A[i]$ and $N[i]$ is added exactly once. 
The total running time of the above process is $\Theta(n)$. 
\\

Now we have all $A[i]$ and $N[i]$, the next step is to find the list ``selectedVertices''. 
This can be done by running a ``check(i)'' function: 
first compare $A[i]$ and $N[i]$. 
If $A[i]$ is greater than $N[i]$, add $i$ to ``selectedVertices'' and check all its grand children; 
else, check all its children. 
\\
We can prove that after running ``check(i)'', 
total weight of selected vertices in $G_i$ is $A[i]$.
We can prove this by induction. 
For a vertex $u_i$ without any child, running ``check(i)'' 
adds itself to ``slectedVertices'' list, which makes sure that $A[i]$ is reached. 
For any $u_i$, if $A[i]$ is greater than $N[i]$, 
we know that $A[i]$ is calculated from 
$$A[i] = p_i + \sum_{u_j is child of u_i}N[j]$$
By running ``check'' on every grandchild $u_k$ of $u_i$, 
$A[k]$ is reached in $G_k$ for all $u_k$. 
Then for every child $u_j$ of $u_i$, the total weight of selected vertices in $G_j$ is 
$N[j]$. 
Then the total weoght of selected vertices in $G_i$ is $A[i]$. 
\\
On the other side, if $A[i] = N[i]$, 
we know that 
$$A[i] = N[i] = \sum_{u_j is child of u_i} A[j]$$
By running ``check'' on every child $u_j$ of $u_i$, 
the total weight of selected vertices in $G_j$ is $A[j]$, and 
then the total weoght of selected vertices in $G_i$ is $A[i]$. 
\\
This step run ``check'' at most $n$ times, and there is only one comparing every running. 
Thus the running time is also $\Theta(n)$. 
Then the total running time is $\Theta(n)$. 



\paragraph{(c)}




Your solution to Problem 1-1 goes here. Remember, \emph{each problem} should be
in a separate \LaTeX\ file so that you can generate one PDF per problem to
submit to Stellar.

\end{document}

