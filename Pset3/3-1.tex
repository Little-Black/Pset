\documentclass{6046}


\author{Lingfu Zhang}
\problem{3-1}
\collab{none}

\begin{document}

\paragraph{(a)}
Based on CRLS, 
all changes we need to make here 
are redefine `high($x$)', 
`low($x$)' and 
index($x$, $y$) as: 
\\

high($x$) = $\lfloor{x/
2^{\lfloor{(\lg{u})/3}\rfloor}
}\rfloor$ 
\\
low($x$) = $x$ mod 
$2^{\lfloor{(\lg{u})/3}\rfloor}$
\\
index($x$, $y$) = $x  
\cdot 2^{\lfloor{(\lg{u})/3}\rfloor}
+ y$
\\
and everything else is exactly the same. 
\\

Now let's consider the consider the 
costs of INSERT, DELETE and SUCCESSOR. 
\\
For INSERT, 
when the cluster $x$ belongs to 
doesn't exist, we do  
vEB-TREE-INSERT($V.summary$, high($x$)), 
which takes 
$T(2^{\lfloor{(\lg{u})/3}\rfloor})$
time; 
when the cluster $x$ belongs to 
exists before, we do 
vEB-TREE-INSERT($V.cluster$[high($x$)], low($x$)), 
which takes 
$T(2^{\lceil{(2\lg{u})/3}\rceil})$
time.  
All other operations take constant time. 
Then we have: 
$$T(u) \leq 
T(2^{\lceil{(2\lg{u})/3}\rceil}) + 
O(1)$$
or 
$$
T(2^m) \leq
T(2^{\lceil{2m/3}\rceil}) + O(1)
\leq
T(2^{3m/4}) + O(1)
$$
for all $m\geq 3$. 
Let $S(m) = T(2^m)$, 
we can get 
$T(u) = T(2^m) = S(m) = O(\lg{m}) = O(\lg{\lg{m}})$. 
\\

For vEB-TREE-DELETE($V$, $x$), 
if the cluster containing $x$ isn't 
empty after 
vEB-TREE-DELETE($V.cluster$[high($x$)], low($x$)), 
we needn't do 
vEB-TREE-DELETE($V.summary$, high($x$)); 
otherwise, 
$V.cluster$[high($x$)] only contains 
$x$ before the deletion, 
and 
\\
vEB-TREE-DELETE($V.cluster$[high($x$)], low($x$)) 
takes constant time. 
So the costs is the same as INSERT. 
\\

For vEB-TREE-SUCCESSOR($V$, $x$), 
we also either do 
vEB-TREE-SUCCESSOR($V.cluster$[high($x$)], low($x$)) 
or
vEB-TREE-SUCCESSOR($V.summary$, high($x$)), 
and some other operations with constant time. 
Thus the total cost is still the same as INSERT. 
\\

Although the costs is still $O(\lg{\lg{u}})$, 
we may state that the costs is higher than before, 
as we now have $S(m) \leq S(3m/4) + O(1)$ 
instead of $S(m) \leq S(2m/3) + O(1)$. 

\paragraph{(b)}
All the following changes are based on the 
pseudocode in CRLS. 
\\

For vEB-TREE-INSERT($V$, $x$), 
first, check whether $V$ is empty: 
if so, do vEB-EMPTY-TREE-INSERT($V$, $x$), 
which keeps the same; 
otherwise, 
if $x < V.min$, swap $x$ with $V.min$; 
if $x > V.max$, swap $x$ with $V.max$. 
Now if $V.u > 2$,  we insert $x$ to lower level vEB structures. 
This is the same as line 5-9 in CRLS.  
\\

All changes we make here takes constant time. 
We now check whether $x > V.max$ at the beginning instead of in the end, 
and make swap if necessary. 
The analysis of costs should be the same, 
and total costs is still $\lg{\lg{u}}$.
\\

For vEB-TREE-DELETE($V$, $x$), 
if $V.min == V.max$, there is only one element in $V$, 
let $V.min = NIL$, $V.max = NIL$. 
Else, if $V.u == 2$, 
let $V.min = 1$ if $x == 0$ and 
$V.max = 0$ if $x == 1$. 
Now consider $V.u > 2$. 
If $x == V.min$, we find 
the next smallest value in $V$, 
and let both $x$ and $V.min$ be that value. 
If $x == V.max$, we find 
the next largest value in $V$, 
and let both $x$ and $V.max$ be that value. 
Now we delete $x$ from lower level vEB structures 
and delete the cluster in $V.summary$ if it becomes empty.  
This is the same as line 13-15 in CLRS. 
\\

All changes made here takes constant time in every recursive call, 
and we deal with the situation where 
$x=V.max$ differently. 
The analysis of costs is basicly the same as 
before. 
The total costs is still $\lg{\lg{u}}$. 
\\

For vEB-TREE-SUCCESSOR($V$, $x$), 
first we deal with the condition where 
$V.u=2$, and this is line 1-4 in CLRS.
Then if $V.min$ is not NIL and $x < V.min$, 
return $V.min$.   
Then try to find successor in $V.cluster$[high($x$)], 
which is the same as line 7-10 in CLRS. 
If $V.cluster$[high($x$)] is empty or low($x$) 
is not less than the maximum value in the cluster, 
find the successor 
of high($x$) in $V.summary$. 
If such a successor exists, return the minimum value of the cluster; 
otherwise, compare $x$ with $V.max$: 
if $x < V.max$, return $V.max$; 
otherwise (including $V.max$ is NIL), return NIL. 
\\

The only change to this is to 
compare $x$ with $V.max$ if all attemps to 
find a successor before fails. 
This takes constant costs, and the total costs 
of SUCCESSOR is still $\lg{\lg{u}}$. 

\end{document}

